\chapter{Conclusão}

Neste trabalho, buscou-se caracterizar um amplificador valvulado de guitarra criando um modelo virtual que possuísse as características principais do amplificador utilizado como molde. Para isso utilizou-se o método de caracterização de sistemas não lineares através do uso de Séries de Volterra. A identificação dos \kernels da Série de Volterra para a modelagem do sistema utilizando o modelo de \textit{Hammerstein}, foi feita com a aplicação de um sinal senoidal variando exponencialmente na frequência, um \textit{Swept Sine} explicado na seção \ref{Sweptsines}. Ao aplicar o \textit{Swept Sine} no amplificador e gravar sua saída, é obtido um \textit{Swep Sine} distorcido. A obtenção dos \kernels é feita pela convolução do \textit{Swept Sine} invertido no tempo, levando ao resultado mostrado na figura \ref{fig:10sweepkernels}. Cada \textit{kernel} é separado em impulsos, que posteriormente são transformados em filtros, conforme a figura \ref{fig:hammer} mostra. A reconstrução da saída através do modelo de \textit{Hammerstein} é feita elevando o sinal de entrada conforme a quantidade de \kernels utilizados na montagem do modelo. A acurácia do sistema foi medida através da aplicação de uma senoide pura com frequência de 1 kHz, comparando o sistema o modelo virtual com o amplificador molde. As respostas gravadas, tiveram os harmônicos medidos para permitir uma visualização numérica entre os modelos, os resultados foram contabilizados na tabela \ref{tab01}.

O modelo apresenta um bom desempenho na construção de um sinal senoidal distorcido, os harmônicos são construídos de forma semelhante ao modelo físico. A figura \ref{fig:tccfig} mostra que existem harmônicos que não foram emulados devido a ordem escolhida do sistema, apenas 11 harmônicos foram emulados. Como os harmônicos perdem intensidade com o aumento da frequência, a influência daqueles com elevadas frequências é pequena na construção do sinal.

A aplicação do modelo em um sistema que funcione em tempo real ainda é um pouco dificultada pelo gasto computacional para processar todo o método de aplicação do sistema de \textit{Hammerstein}, inviabilizando um sistema de baixa latência.

A utilização de redes neurais aparenta ser uma solução para a criação de um modelo fiel a modelagem de sistemas não lineares, podendo desenvolver aplicações operando em tempo real com baixa latência. Arquiteturas implementadas com redes neurais podem ser treinadas com um conjunto de dados que fornece a resposta frequencial do amplificador utilizado como molde.